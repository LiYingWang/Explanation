% Template for PLoS

\documentclass[10pt]{article}

% amsmath package, useful for mathematical formulas
\usepackage{amsmath}
% amssymb package, useful for mathematical symbols
\usepackage{amssymb}

% hyperref package, useful for hyperlinks
\usepackage{hyperref}

% graphicx package, useful for including eps and pdf graphics
% include graphics with the command \includegraphics
\usepackage{graphicx}

% Sweave(-like)
\usepackage{fancyvrb}
\DefineVerbatimEnvironment{Sinput}{Verbatim}{fontshape=sl}
\DefineVerbatimEnvironment{Soutput}{Verbatim}{}
\DefineVerbatimEnvironment{Scode}{Verbatim}{fontshape=sl}
\newenvironment{Schunk}{}{}
\DefineVerbatimEnvironment{Code}{Verbatim}{}
\DefineVerbatimEnvironment{CodeInput}{Verbatim}{fontshape=sl}
\DefineVerbatimEnvironment{CodeOutput}{Verbatim}{}
\newenvironment{CodeChunk}{}{}

% cite package, to clean up citations in the main text. Do not remove.
\usepackage{cite}

\usepackage{color}

% Use doublespacing - comment out for single spacing
%\usepackage{setspace}
%\doublespacing


% Text layout
\topmargin 0.0cm
\oddsidemargin 0.5cm
\evensidemargin 0.5cm
\textwidth 16cm
\textheight 21cm

% Bold the 'Figure #' in the caption and separate it with a period
% Captions will be left justified
\usepackage[labelfont=bf,labelsep=period,justification=raggedright]{caption}

% Use the PLoS provided bibtex style
\bibliographystyle{plos}

% Remove brackets from numbering in List of References
\makeatletter
\renewcommand{\@biblabel}[1]{\quad#1.}
\makeatother


% Leave date blank
\date{}

\pagestyle{myheadings}
%% ** EDIT HERE **


%% ** EDIT HERE **
%% PLEASE INCLUDE ALL MACROS BELOW

%% END MACROS SECTION


\begin{document}

% Title must be 150 characters or less
\begin{flushleft}
{\Large
\textbf{A Capitalized Title: Something about a great Discovery}
}
% Insert Author names, affiliations and corresponding author email.
\\
  Liying Wang\textsuperscript{1,2*}\\
\bf{1} Department/Anthropology/Center, Institution Name,  Seattle,  Washington,  USA
\\
\bf{2} Dept/Program/Center, Institution Name,  City,  State,  Country
\\

\textasteriskcentered{} E-mail:   \href{mailto:liying15@uw.edu}{\nolinkurl{liying15@uw.edu}}

\end{flushleft}

\subsubsection{Introduction}\label{introduction}

In this paper, I will focus on a debate about whether the separate
cultural layers in Kiwulan (KWL) site belong to the same prehistoric
ethnic group. The KWL site is located at Ilan city and near a riverside
at the northern margin of the Ilan plain. The site was excavated during
2001 to 2003 by the Department of Anthropology of National Taiwan
University (Chen 2007). According to the archaeological remains, the
site can be divided into a lower cultural layer and an upper cultural
layer, dating from 1300B.P. to 800B.P. and 600B.P. to 100B.P.,
respectively, based on radiocarbon dates. However, there is a debate
about whether the archaeological remains from both layers belong to the
same culture or not. Chen (2004) argues that these two layers belong to
the same culture based on similar pattern of artifacts. However, Chiu
states that they belong to different culture or ethnic group due to the
distinct style of mortuary practice. The following will further examine
these two explanations.

\subsubsection{Archaeological evidence}\label{archaeological-evidence}

Chen focuses on the similar archaeological remains in both layers. In
both layer, we can find similar burial goods, such as glass beads, agate
beads, and metal bell. In addition, the pattern of pottery in both
layers shows the similar surface treatment, stamped decoration. Chiu
stresses the difference between these two layers based on mortuary
practice. The common burial in lower cultural layer is secondary;
however, most burials in upper cultural layer are primary and the bodies
are in flexed position.

\subsubsection{Links between evidence and
behavior}\label{links-between-evidence-and-behavior}

The same culture hypothesis is based on the similar human practice on
daily life, such as making pattery. Different culture hypothesis thinks
that mortury parctice reflects the worldview of a ethnic group, which is
the core element in a cultue and seldom change with time. \#\#\#Behavior
at different scales

\subsubsection{Discussion}\label{discussion}

\begin{enumerate}
\def\labelenumi{\arabic{enumi}.}
\itemsep1pt\parskip0pt\parsep0pt
\item
  Context of the monograph's explanatory model
\item
  Relevant philosophy of science
\item
  Critique
\end{enumerate}

\subsubsection{Conclusion}\label{conclusion}

\subsubsection{Reference}\label{reference}

Chiu, Hung-Lin 2004 Investigations of Mortuary Behaviors and Cultural
Change of the Kivulan Site in I-Lan County, Taiwan.
宜蘭縣礁溪鄉淇武蘭遺址出土墓葬研究---埋葬行為與文化變遷的觀察,
Department of Anthropology, National Taiwan University.

Chen, Yu-Pei 2007 The Excavation Report of the Ki-Wu-Lan Site 6. I-lan,
Taiwan: Lanyang museum.

2004 The significance of the Kiwulan site for understahnding the
prehistoric period in Ilan Plain.
淇武蘭遺址發掘對蘭陽平原史前研究的意義. Ilan Study.
宜蘭研究第六屆學術研討會, 宜蘭: 宜蘭縣史館 6.

Lin, Shu-fen 2004 Environmental and Climatic Changes in Ilan Plain over
the Recent 4200 Years as Revealed by Pollen Data and their Relationship
to Prehistory
Colonization由孢粉紀錄看宜蘭平原最近4200年來的自然環境演變及其與史前文化發展之關係,
Department of Geosciences, National Taiwan University.

\subsubsection{Comment}\label{comment}

Search other sites, when burial costom changes inferecne of (IBE) Yayoin
Jomon different part of explanation in a same thesis.

\end{document}

